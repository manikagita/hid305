\documentclass[sigconf]{acmart}

\usepackage{hyperref}

\usepackage{endfloat}
\renewcommand{\efloatseparator}{\mbox{}} % no new page between figures

\usepackage{booktabs} % For formal tables

\settopmatter{printacmref=false} % Removes citation information below abstract
\renewcommand\footnotetextcopyrightpermission[1]{} % removes footnote with conference information in first column
\pagestyle{plain} % removes running headers

\begin{document}
\title{Big Data Analytics for Municipal Waste Management}

\author{Andres Castro Benavides}
\orcid{1234-5678-9012}
\affiliation{%
  \institution{Indiana University}
  \streetaddress{107 S. Indiana Avenue}
  \city{Bloomington} 
  \state{Indiana} 
  \postcode{43017-6221}
}
\email{acastrob@iu.edu}

\author{Mani Kaguita}
\affiliation{%
  \institution{Indiana University}
  \streetaddress{P.O. Box 1212}
  \city{Dublin} 
  \state{Ohio} 
  \postcode{43017-6221}
}
\email{mkaguita@edu.iu}


% The default list of authors is too long for headers}
\renewcommand{\shortauthors}{B. Trovato et al.}


\begin{abstract}
As waste management becomes a greater concern for cities and municipalities around the world, big data analysis has the potential to not only help assess the current waste management strategies, but also provide information that can be used to optimize the systems used in various institutions, local government, companies, etc.

\end{abstract}

\keywords{Waste Management, Big Data, Local Government}



\maketitle

\section{Introduction}

Concept of waste management…

Solid Waste Management (SWM) is a set of consistent and systematic regulations related to control generation, storage, collection, transportation, processing and land filling of wastes according to the best public health principles, economy, preservation of resources, aesthetics, other environmental requirements and what the public attends to ~\cite{akbarpour2016}

Managing solid waste is one of the most essential services which often fails due to
rapid urbanization along with changes in the waste quantity and composition.
Quantity and composition vary from country to country making them difficult to
adopt for waste management system which may be successful at other places.
Quantity and composition of solid waste vary from place to place ~\cite{chandrappa2012} 

%maybe we should talk about big data here as well…. 



\section{Opportunities for Waste Management Optimization}

By collecting and storing data related to types of waste, quantities, periodicity and  composition.

\subsection{GIS Analytics}


\section{Statistics and Waste Management}




While rural area usually generates organic and biodegradable, urban area produces waste influenced by culture and practices of society.~\cite{chandrappa2012} p47 to 63



There are many data analysis methods that are used when studying waste management, but the two most popular are PCA and PLS1. 
~\cite{bohm2013}



decision makers should distinguish between optimal, good, and fortuitous decision-making. In the optimal decision making, one can solve the optimal problem using the techniques available in other fields. In this solution method, generally some constraints (criteria) are consid-
ered, where the function(s) is to be optimized through applying some methods. Good decision-making is done based on experience, trial and error or comparison between different options of the integrated SWM.
Although it is possible to choose decisions close to the optimal state using this decision-making method, today these methods are not applicable due to increased number of different combinations in the decision-making process. In the fortuitous decision-making, since decisions are made with no scientific base, so the results are not acceptable ~\cite{akbarpour2016}

The process of solving a math program requires a large number of calculations and is, therefore, best performed by a computer program. Lingo is a mathematical model-
ing language designed particularly for formulating and solving a wide variety of optimization problems including linear programing. Lingo optimization software uses branch and bound methods to solve problems of this type. ~\cite{akbarpour2016}



\section{Conclusions}

Working on this



\appendix


Generated by bibtex from your \texttt{.bib} file.  Run latex, then
bibtex, then latex twice (to resolve references) to create the
\texttt{.bbl} file.  Insert that \texttt{.bbl} file into the
\texttt{.tex} source file and comment out the command
\texttt{{\char'134}thebibliography}.

% This next section command marks the start of
% Appendix B, and does not continue the present hierarchy

\section{More Help for the Hardy}

Of course, reading the source code is always useful.  The file
\path{acmart.pdf} contains both the user guide and the commented code.

\begin{acks}

  The authors would like to thank Dr. Yuhua Li for providing the
  matlab code of the \textit{BEPS} method.

  The authors would also like to thank the anonymous referees for
  their valuable comments and helpful suggestions. The work is
  supported by the \grantsponsor{GS501100001809}{National Natural
    Science Foundation of
    China}{http://dx.doi.org/10.13039/501100001809} under Grant
  No.:~\grantnum{GS501100001809}{61273304}
  and~\grantnum[http://www.nnsf.cn/youngscientsts]{GS501100001809}{Young
    Scientsts' Support Program}.

\end{acks}

\bibliographystyle{ACM-Reference-Format}
\bibliography{report1} 

\end{document}
